\documentclass[titlepage,11pt,a4paper]{article}

\usepackage{polytechnique}

\title{Rapport de MODAL}
\subtitle{Comment faire suivre les murs � un dr�ne}

\author{Basile \textsc{Bruneau}\\
Youssef \textsc{Achari-Berrada}}

\date{Juin 2015}

\begin{document}
\maketitle



\part{Strat�gie}
L'AR.Drone disposant de peu de capteurs (pas de GPS, pas de cam�ra de profondeur) le seul que nous pouvions utiliser pour r�aliser ce projet est la cam�ra frontale. C'est uniquement � partir de cet unique flux vid�o que nous avons essay� de faire suivre les murs au drone.

L'objectif �tant de d�placer le drone, c'est le mouvement ici qui nous int�resse. Nous avons donc d�cid� d'utiliser le flux optique pour obtenir le mouvement des points de l'image. Nous avons repris l'algorithme lkTracks qui calcule uniquement le mouvement des coins d�tect�s dans l'image (ce qui permet un traitement rapide, contrairement � un calcul du mouvement de tous les points de l'image). Comme nous le verrons plus tard, l'inconv�nient est que les situation o� aucun coin n'est d�tect� dans l'image ne sont pas rares.

Ensuite l'observation importante est que lorsque le robot vole et avance en longeant un mur � sa droite (en regardant devant lui) alors les points de l'image les plus � droite ont une vitesse plus �lev�e que les points qui sont plus vers le centre de l'image. Et lorsque le drone avance � vitesse constante, le profil des vitesse doit probablement rester constant : les points � droite auront toujours la m�me vitesse, et ceux plus vers le centre �galement. Nous avons donc essay� de trouver ce profil des vitesses lorsque le drone longe un mur � une certaine vitesse, et ensuite en comparant la vitesse des points avec le profil th�orique, nous allons essayer de corriger la trajectoire du robot.

Nous avons donc pos� l'AR.Drone sur un TurtleBot que nous avons fait progress� en ligne droite le long d'un mur. Puis nous avons enregistr� toutes les vitesses des points d�tect�s par la cam�ra du drone. Afin d'avoir un maximum d'informations, nous avons ajout� des coins nous m�me sur le mur. Nous avons r�p�t� l'exp�rience une dizaine de fois, ce qui nous a permis d'obtenir les vitesses d'environ 25 000 points.

\end{document}